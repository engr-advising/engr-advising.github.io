%% Generated by Sphinx.
\def\sphinxdocclass{report}
\documentclass[letterpaper,10pt,english]{sphinxmanual}
\ifdefined\pdfpxdimen
   \let\sphinxpxdimen\pdfpxdimen\else\newdimen\sphinxpxdimen
\fi \sphinxpxdimen=.75bp\relax
\ifdefined\pdfimageresolution
    \pdfimageresolution= \numexpr \dimexpr1in\relax/\sphinxpxdimen\relax
\fi
%% let collapsable pdf bookmarks panel have high depth per default
\PassOptionsToPackage{bookmarksdepth=5}{hyperref}

\PassOptionsToPackage{warn}{textcomp}
\usepackage[utf8]{inputenc}
\ifdefined\DeclareUnicodeCharacter
% support both utf8 and utf8x syntaxes
  \ifdefined\DeclareUnicodeCharacterAsOptional
    \def\sphinxDUC#1{\DeclareUnicodeCharacter{"#1}}
  \else
    \let\sphinxDUC\DeclareUnicodeCharacter
  \fi
  \sphinxDUC{00A0}{\nobreakspace}
  \sphinxDUC{2500}{\sphinxunichar{2500}}
  \sphinxDUC{2502}{\sphinxunichar{2502}}
  \sphinxDUC{2514}{\sphinxunichar{2514}}
  \sphinxDUC{251C}{\sphinxunichar{251C}}
  \sphinxDUC{2572}{\textbackslash}
\fi
\usepackage{cmap}
\usepackage[T1]{fontenc}
\usepackage{amsmath,amssymb,amstext}
\usepackage{babel}



\usepackage{tgtermes}
\usepackage{tgheros}
\renewcommand{\ttdefault}{txtt}



\usepackage[Bjarne]{fncychap}
\usepackage{sphinx}

\fvset{fontsize=auto}
\usepackage{geometry}


% Include hyperref last.
\usepackage{hyperref}
% Fix anchor placement for figures with captions.
\usepackage{hypcap}% it must be loaded after hyperref.
% Set up styles of URL: it should be placed after hyperref.
\urlstyle{same}

\addto\captionsenglish{\renewcommand{\contentsname}{Contents:}}

\usepackage{sphinxmessages}
\setcounter{tocdepth}{0}



\title{NSLS-II SRX Documentation}
\date{Jun 16, 2021}
\release{}
\author{SRX}
\newcommand{\sphinxlogo}{\vbox{}}
\renewcommand{\releasename}{}
\makeindex
\begin{document}

\pagestyle{empty}
\sphinxmaketitle
\pagestyle{plain}
\sphinxtableofcontents
\pagestyle{normal}
\phantomsection\label{\detokenize{index::doc}}



\chapter{Introduction to SRX}
\label{\detokenize{intro:introduction-to-srx}}\label{\detokenize{intro::doc}}
\sphinxAtStartPar
The Sub\sphinxhyphen{}micron Resolution X\sphinxhyphen{}ray Spectroscopy (SRX) beamline at the NSLS\sphinxhyphen{}II supports a wide variety of scientific use\sphinxhyphen{}cases, ranging from geoscience through energy materials. Facilities include state\sphinxhyphen{}of\sphinxhyphen{}the\sphinxhyphen{}art, sub\sphinxhyphen{}micron focusing X\sphinxhyphen{}ray optics and a flexible sample environment.

\begin{figure}[htbp]
\centering
\capstart

\sphinxhref{\_images/SRX\_BeamlineSchematic\_text\_tab\_web.png}{\sphinxincludegraphics[width=1.000\linewidth]{{SRX_BeamlineSchematic_text_tab_web}.png}}
\caption{A schematic of the SRX beamline at the NSLS\sphinxhyphen{}II.}\label{\detokenize{intro:id1}}\label{\detokenize{intro:fig-prompt}}\end{figure}


\chapter{Useful Commands}
\label{\detokenize{useful:useful-commands}}\label{\detokenize{useful::doc}}
\sphinxAtStartPar
Below is a list of useful commands for running the SRX beamline. Previous commands can be seen by hitting the up arrow in Bluesky. To search through them, you can start typing a command before hitting the up arrow to filter your history.


\section{Starting Bluesky}
\label{\detokenize{useful:starting-bluesky}}
\sphinxAtStartPar
Start Bluesky \sphinxhyphen{} \sphinxstyleemphasis{Start Bluesky from the terminal.}

\begin{sphinxVerbatim}[commandchars=\\\{\}]
\PYGZdl{} bsui
\end{sphinxVerbatim}


\section{General Functions}
\label{\detokenize{useful:general-functions}}
\sphinxAtStartPar
Change X\sphinxhyphen{}ray energy \sphinxhyphen{} \sphinxstyleemphasis{Either command can be used below. The energy can be entered in units of eV or keV.}

\begin{sphinxVerbatim}[commandchars=\\\{\}]
\PYG{n}{Bluesky}\PYG{n+nd}{@SRX} \PYG{p}{[}\PYG{l+m+mi}{1}\PYG{p}{]} \PYG{n}{energy}\PYG{o}{.}\PYG{n}{move}\PYG{p}{(}\PYG{l+m+mf}{7.2}\PYG{p}{)}
\PYG{n}{Bluesky}\PYG{n+nd}{@SRX} \PYG{p}{[}\PYG{l+m+mi}{2}\PYG{p}{]} \PYG{o}{\PYGZpc{}}\PYG{n}{mov} \PYG{n}{energy} \PYG{l+m+mf}{7.2}
\end{sphinxVerbatim}

\sphinxAtStartPar
Optimize the beam \sphinxhyphen{} \sphinxstyleemphasis{Maximize the X\sphinxhyphen{}ray flux.}

\begin{sphinxVerbatim}[commandchars=\\\{\}]
\PYG{n}{RE}\PYG{p}{(}\PYG{n}{peakup\PYGZus{}fine}\PYG{p}{(}\PYG{p}{)}\PYG{p}{)}
\end{sphinxVerbatim}

\sphinxAtStartPar
Setting a region of interest \sphinxhyphen{} \sphinxstyleemphasis{Set the ROI on the detector. The specific edge is optional.}

\begin{sphinxVerbatim}[commandchars=\\\{\}]
\PYG{n}{Bluesky}\PYG{n+nd}{@SRX} \PYG{p}{[}\PYG{l+m+mi}{1}\PYG{p}{]} \PYG{n}{setroi}\PYG{p}{(}\PYG{l+m+mi}{1}\PYG{p}{,} \PYG{l+s+s1}{\PYGZsq{}}\PYG{l+s+s1}{Fe}\PYG{l+s+s1}{\PYGZsq{}}\PYG{p}{)}
\PYG{n}{Bluesky}\PYG{n+nd}{@SRX} \PYG{p}{[}\PYG{l+m+mi}{2}\PYG{p}{]} \PYG{n}{setroi}\PYG{p}{(}\PYG{l+m+mi}{1}\PYG{p}{,} \PYG{l+s+s1}{\PYGZsq{}}\PYG{l+s+s1}{Fe}\PYG{l+s+s1}{\PYGZsq{}}\PYG{p}{,} \PYG{l+s+s1}{\PYGZsq{}}\PYG{l+s+s1}{ka1}\PYG{l+s+s1}{\PYGZsq{}}\PYG{p}{)}
\end{sphinxVerbatim}


\section{XRF Imaging}
\label{\detokenize{useful:xrf-imaging}}
\sphinxAtStartPar
Fly scan \sphinxhyphen{} \sphinxstyleemphasis{Perform a fly scan. Return an image with dimensions (numX, numY)}

\begin{sphinxVerbatim}[commandchars=\\\{\}]
\PYG{n}{Bluesky}\PYG{n+nd}{@SRX} \PYG{p}{[}\PYG{l+m+mi}{1}\PYG{p}{]} \PYG{n}{RE}\PYG{p}{(}\PYG{n}{nano\PYGZus{}scan\PYGZus{}and\PYGZus{}fly}\PYG{p}{(}\PYG{n}{startX}\PYG{p}{,} \PYG{n}{stopX}\PYG{p}{,} \PYG{n}{numX}\PYG{p}{,}
                                     \PYG{n}{startY}\PYG{p}{,} \PYG{n}{stopY}\PYG{p}{,} \PYG{n}{numY}\PYG{p}{,} \PYG{n}{dwell}\PYG{p}{)}\PYG{p}{)}
\PYG{n}{Bluesky}\PYG{n+nd}{@SRX} \PYG{p}{[}\PYG{l+m+mi}{2}\PYG{p}{]} \PYG{n}{RE}\PYG{p}{(}\PYG{n}{nano\PYGZus{}y\PYGZus{}scan\PYGZus{}and\PYGZus{}fly}\PYG{p}{(}\PYG{n}{startY}\PYG{p}{,} \PYG{n}{stopY}\PYG{p}{,} \PYG{n}{numY}\PYG{p}{,}
                                       \PYG{n}{startX}\PYG{p}{,} \PYG{n}{stopX}\PYG{p}{,} \PYG{n}{numX}\PYG{p}{,} \PYG{n}{dwell}\PYG{p}{)}\PYG{p}{)}
\end{sphinxVerbatim}

\sphinxAtStartPar
Step scan \sphinxhyphen{} \sphinxstyleemphasis{Perform a step scan. Note: these arguments take a step size, not the number of points.}

\begin{sphinxVerbatim}[commandchars=\\\{\}]
\PYG{n}{Bluesky}\PYG{n+nd}{@SRX} \PYG{p}{[}\PYG{l+m+mi}{1}\PYG{p}{]} \PYG{n}{RE}\PYG{p}{(}\PYG{n}{nano\PYGZus{}xrf}\PYG{p}{(}\PYG{n}{startX}\PYG{p}{,} \PYG{n}{stopX}\PYG{p}{,} \PYG{n}{stepX}\PYG{p}{,}
                            \PYG{n}{startY}\PYG{p}{,} \PYG{n}{stopY}\PYG{p}{,} \PYG{n}{stepY}\PYG{p}{,} \PYG{n}{dwell}\PYG{p}{)}\PYG{p}{)}
\end{sphinxVerbatim}


\section{XAS Spectroscopy}
\label{\detokenize{useful:xas-spectroscopy}}
\sphinxAtStartPar
Print element binding energies \sphinxhyphen{} \sphinxstyleemphasis{Print the binding energies for the element of interest. The “best” edge can be returned as available.}

\begin{sphinxVerbatim}[commandchars=\\\{\}]
\PYG{n}{Bluesky}\PYG{n+nd}{@SRX} \PYG{p}{[}\PYG{l+m+mi}{1}\PYG{p}{]} \PYG{n}{Fe\PYGZus{}k} \PYG{o}{=} \PYG{n}{getbindingE}\PYG{p}{(}\PYG{l+s+s1}{\PYGZsq{}}\PYG{l+s+s1}{Fe}\PYG{l+s+s1}{\PYGZsq{}}\PYG{p}{)}
\end{sphinxVerbatim}

\sphinxAtStartPar
Print element emission energies \sphinxhyphen{} \sphinxstyleemphasis{Print the emission energies for the element of interest.}

\begin{sphinxVerbatim}[commandchars=\\\{\}]
\PYG{n}{Bluesky}\PYG{n+nd}{@SRX} \PYG{p}{[}\PYG{l+m+mi}{1}\PYG{p}{]} \PYG{n}{getemissionE}\PYG{p}{(}\PYG{l+s+s1}{\PYGZsq{}}\PYG{l+s+s1}{Fe}\PYG{l+s+s1}{\PYGZsq{}}\PYG{p}{)}
\end{sphinxVerbatim}

\sphinxAtStartPar
XANES scan \sphinxhyphen{} \sphinxstyleemphasis{Run a XANES scan. This scan has 3 regions with different steps spanning the iron K\sphinxhyphen{}edge.}

\begin{sphinxVerbatim}[commandchars=\\\{\}]
\PYG{n}{Bluesky}\PYG{n+nd}{@SRX} \PYG{p}{[}\PYG{l+m+mi}{1}\PYG{p}{]} \PYG{n}{RE}\PYG{p}{(}\PYG{n}{xanes\PYGZus{}plan}\PYG{p}{(}\PYG{n}{erange}\PYG{o}{=}\PYG{p}{[}\PYG{n}{Fe\PYGZus{}k}\PYG{o}{\PYGZhy{}}\PYG{l+m+mi}{50}\PYG{p}{,} \PYG{n}{Fe\PYGZus{}k}\PYG{o}{\PYGZhy{}}\PYG{l+m+mi}{10}\PYG{p}{,} \PYG{n}{Fe\PYGZus{}k}\PYG{o}{+}\PYG{l+m+mi}{50}\PYG{p}{,} \PYG{n}{Fe\PYGZus{}k}\PYG{o}{+}\PYG{l+m+mi}{150}\PYG{p}{]}\PYG{p}{,}
                              \PYG{n}{estep}\PYG{o}{=}\PYG{p}{[}\PYG{l+m+mf}{2.0}\PYG{p}{,} \PYG{l+m+mf}{1.0}\PYG{p}{,} \PYG{l+m+mf}{2.0}\PYG{p}{]}\PYG{p}{,}
                              \PYG{n}{acqtime}\PYG{o}{=}\PYG{l+m+mf}{1.0}\PYG{p}{,}
                              \PYG{n}{samplename}\PYG{o}{=}\PYG{l+s+s1}{\PYGZsq{}}\PYG{l+s+s1}{Fe foil}\PYG{l+s+s1}{\PYGZsq{}}\PYG{p}{,}
                              \PYG{n}{filename}\PYG{o}{=}\PYG{l+s+s1}{\PYGZsq{}}\PYG{l+s+s1}{Fe\PYGZus{}foil}\PYG{l+s+s1}{\PYGZsq{}}\PYG{p}{]}\PYG{p}{)}\PYG{p}{)}
\end{sphinxVerbatim}


\section{Troubleshooting}
\label{\detokenize{useful:troubleshooting}}
\sphinxAtStartPar
Pause a scan \sphinxhyphen{} \sphinxstyleemphasis{The scan will pause at the next checkpoint.}

\begin{sphinxVerbatim}[commandchars=\\\{\}]
\PYG{n}{CTRL}\PYG{o}{+}\PYG{n}{C}
\end{sphinxVerbatim}

\sphinxAtStartPar
Urgently stop a scan \sphinxhyphen{} \sphinxstyleemphasis{With each CTRL\sphinxhyphen{}C, Bluesky raisens the urgency of stopping the scan.}

\begin{sphinxVerbatim}[commandchars=\\\{\}]
\PYG{n}{CTRL}\PYG{o}{+}\PYG{n}{C} \PYG{n}{x20}
\end{sphinxVerbatim}

\sphinxAtStartPar
Resume a scan \sphinxhyphen{} \sphinxstyleemphasis{A scan can be resumed after pausing.}

\begin{sphinxVerbatim}[commandchars=\\\{\}]
\PYG{n}{Bluesky}\PYG{n+nd}{@SRX} \PYG{p}{[}\PYG{l+m+mi}{1}\PYG{p}{]} \PYG{n}{RE}\PYG{o}{.}\PYG{n}{resume}\PYG{p}{(}\PYG{p}{)}
\end{sphinxVerbatim}

\sphinxAtStartPar
Stop a scan \sphinxhyphen{} \sphinxstyleemphasis{Stop a scan and label the scan as a success or failure.}

\begin{sphinxVerbatim}[commandchars=\\\{\}]
\PYG{n}{Bluesky}\PYG{n+nd}{@SRX} \PYG{p}{[}\PYG{l+m+mi}{1}\PYG{p}{]} \PYG{n}{RE}\PYG{o}{.}\PYG{n}{stop}\PYG{p}{(}\PYG{p}{)}   \PYG{c+c1}{\PYGZsh{} Label scan as success}
\PYG{n}{Bluesky}\PYG{n+nd}{@SRX} \PYG{p}{[}\PYG{l+m+mi}{2}\PYG{p}{]} \PYG{n}{RE}\PYG{o}{.}\PYG{n}{abort}\PYG{p}{(}\PYG{p}{)}  \PYG{c+c1}{\PYGZsh{} Label scan as failure}
\end{sphinxVerbatim}


\chapter{Beamline Staff Pages {[}staff only{]}}
\label{\detokenize{staff:beamline-staff-pages-staff-only}}\label{\detokenize{staff::doc}}
\sphinxAtStartPar
The information provided on this page is directed towards helping beamline staff.


\section{Quick Links}
\label{\detokenize{staff:quick-links}}\begin{enumerate}
\sphinxsetlistlabels{\arabic}{enumi}{enumii}{}{.}%
\item {} 
\sphinxAtStartPar
\sphinxhref{http://xf05idd-webcam1.nsls2.bnl.local}{Analog camera web server}

\item {} 
\sphinxAtStartPar
\sphinxhref{http://xf05idd-webcam2.nsls2.bnl.local}{Ceiling camera web server}

\item {} 
\sphinxAtStartPar
\sphinxhref{https://controlsweb.nsls2.bnl.gov/trac/}{Trac ticket system}

\item {} 
\sphinxAtStartPar
\sphinxhref{http://jira.nsls2.bnl.gov}{Jira ticket system}

\item {} 
\sphinxAtStartPar
\sphinxhref{https://nsls2bid.bnl.gov/SAF/Index/5-ID}{SRX Posted SAFs}

\item {} 
\sphinxAtStartPar
\sphinxhref{http://xf05id2-ca1.nsls2.bnl.local/logbook/index.html}{SRX Olog}

\item {} 
\sphinxAtStartPar
\sphinxhref{https://ps.bnl.gov/docs/Reference/NSLS-II\%20Beamline\%205-ID\%20Radiation\%20Safety\%20Component\%20Checklist\%20TEMPLATE.pdf}{SRX Radiation Safety Component Checklist}

\item {} 
\sphinxAtStartPar
\sphinxhref{http://nsls-ii.github.io/}{NSLS\sphinxhyphen{}II Controls Documentation}

\item {} 
\sphinxAtStartPar
\sphinxhref{https://controlsweb01.nsls2.bnl.gov/IP/?page=login\&section=timeout}{NSLS\sphinxhyphen{}II IP Address Management}

\item {} 
\sphinxAtStartPar
\sphinxhref{https://ps.bnl.gov/phot/ros/Shared\%20Documents/MAXIMO\%20Development/Beamlines/LT-R-XFD-CO-DR-SRX-002\_Rev1.xlsx}{NSLS\sphinxhyphen{}II Sharepoint Documentation}

\item {} 
\sphinxAtStartPar
\sphinxhref{https://beamline5id.bnl.gov/index.php/Main\_Page}{Former SRX Wiki Staff Page}

\end{enumerate}

\begin{sphinxadmonition}{note}{\label{\detokenize{staff:id1}}Todo:}\begin{itemize}
\item {} 
\sphinxAtStartPar
Fix IOCs reference

\end{itemize}
\end{sphinxadmonition}


\section{Setting up Users}
\label{\detokenize{staff:setting-up-users}}\begin{enumerate}
\sphinxsetlistlabels{\arabic}{enumi}{enumii}{}{.}%
\item {} 
\sphinxAtStartPar
Post SAF to \sphinxhref{http://passadmin.bnl.gov}{PASS}

\item {} \begin{description}
\item[{Update user\sphinxhyphen{}specific metadata}] \leavevmode\begin{itemize}
\item {} 
\sphinxAtStartPar
Open /nsls2/users/xf05id1/.ipython/profile\_collection/startup/90\sphinxhyphen{}userdata.py

\item {} 
\sphinxAtStartPar
Update proposal dictionary with information from posted SAF. Save.

\item {} 
\sphinxAtStartPar
Restart bluesky.

\end{itemize}

\end{description}

\item {} 
\sphinxAtStartPar
Perform beamline specific training. \sphinxhref{https://www.bnl.gov/ps/training/Beamline-BST-Forms/PS-BST-5-ID.pdf}{5\sphinxhyphen{}ID BST Form}

\end{enumerate}


\section{Beamline Setup}
\label{\detokenize{staff:beamline-setup}}
\sphinxAtStartPar
\sphinxstyleemphasis{These tasks are typically done once a cycle}

\begin{sphinxadmonition}{note}{\label{\detokenize{staff:id2}}Todo:}\begin{itemize}
\item {} 
\sphinxAtStartPar
Beamline alignment

\item {} 
\sphinxAtStartPar
Setting up the Merlin

\item {} 
\sphinxAtStartPar
Setting up the Dexela

\item {} 
\sphinxAtStartPar
Setting up the Xspress3

\end{itemize}
\end{sphinxadmonition}


\subsection{Preparing for a new cycle}
\label{\detokenize{staff:preparing-for-a-new-cycle}}
\sphinxAtStartPar
This is a comprehensive list of things to consider before the start of a cycle.
\begin{enumerate}
\sphinxsetlistlabels{\arabic}{enumi}{enumii}{}{.}%
\item {} 
\sphinxAtStartPar
Close all system safety work permits.

\item {} 
\sphinxAtStartPar
Check cryocooler pressure and load.

\item {} 
\sphinxAtStartPar
Check all vacuum, temperature, water systems.

\item {} 
\sphinxAtStartPar
Check and top\sphinxhyphen{}off the PPS burn\sphinxhyphen{}through.

\item {} 
\sphinxAtStartPar
Confirm the RGA in the A\sphinxhyphen{}hutch is connected and scanning.

\item {} 
\sphinxAtStartPar
Perform a Radiation Safety Component Checklist.

\item {} 
\sphinxAtStartPar
Post a valid SAF and ESR.

\item {} 
\sphinxAtStartPar
Test and deploy the latest bluesky environment.

\item {} 
\sphinxAtStartPar
Setup lsyncd.

\end{enumerate}


\subsection{Aligning the Beamline}
\label{\detokenize{staff:aligning-the-beamline}}\begin{description}
\item[{Historically, the beamline and storage ring take about a day to stabilize. Therefore, on day 1 of operations, it makes sense to open the front\sphinxhyphen{}end shutter and get light through the monochromator. Since components will drift, optimization should take place on day 2 and after a local bump is performed.}] \leavevmode\begin{enumerate}
\sphinxsetlistlabels{\arabic}{enumi}{enumii}{}{.}%
\item {} 
\sphinxAtStartPar
Previous motor positions should be captured at the end of each cycle. As a precaution, capture the current motor positions.

\item {} \begin{description}
\item[{Check front\sphinxhyphen{}end (FE) slits, white\sphinxhyphen{}beam (WB) slits, and mirror (HFM) position.}] \leavevmode\begin{itemize}
\item {} 
\sphinxAtStartPar
Open the gap of the undulator to 18 000 μm. Insert the camera in the HFM tank.

\item {} 
\sphinxAtStartPar
Open the FE shutter. Open the WB slits all the way (4 mm x 4 mm). Turn off mirror pitch feedback and reset the voltage to 30 V. Remove the HFM by dropping the pitch to 0.0 mrad and translating in the positive direction by 3 mm.

\item {} 
\sphinxAtStartPar
Tweak the FE slits so the slits are just protecting the mask.

\item {} 
\sphinxAtStartPar
Tweak the WB slits so the slits are centered. Return the WB slit gaps to their previous values (0.5 mm V x 2.0 mm H).

\item {} 
\sphinxAtStartPar
Bring the HFM back in. Center the mirror on the incoming beam and confirm the mirror is parallel to the beam. Pitch to the nominal 2.5 mrad. Enable the mirror pitch feedback.

\item {} 
\sphinxAtStartPar
Close the FE shutter and retract the camera.

\end{itemize}

\end{description}

\item {} \begin{description}
\item[{Align the monochromator to allow light through.}] \leavevmode\begin{itemize}
\item {} 
\sphinxAtStartPar
With the FE shutter closed, insert the BPM 1 camera (\sphinxstyleemphasis{this can be slow}).

\item {} 
\sphinxAtStartPar
In bluesky, set the energy to the last used value. If starting from scratch, choose a higher energy such as 12 keV. By using bluesky, this will set the undulator gap, and monochomator positions to a reasonably close value.

\item {} 
\sphinxAtStartPar
Open the pink\sphinxhyphen{}beam (PB) slits to a 4.0 mm gap to make sure they are fully open.

\item {} 
\sphinxAtStartPar
Open the FE shutter and hopefully light will come through onto the camera.

\item {} 
\sphinxAtStartPar
If not, set the exposure time on the camera to something large, like 0.1 s. This will help you see the light come through while you scan the motors. There are 4 motors that can be off: Bragg, crystal offset, roll, and pitch. Hopefully, by starting with Bragg you can start to see some light and then optimize by tweaking pitch and roll. Finally, the position of the beam can be translated with the crystal offset.

\item {} 
\sphinxAtStartPar
The PB slits can be centered and closed so they are just intercepting the beam.

\item {} 
\sphinxAtStartPar
Once the light is through the monochromator, the FE shutter can be closed, the camera removed, and alignment downstream can continue.

\end{itemize}

\end{description}

\item {} 
\sphinxAtStartPar
Tweak monochromator and mirror alignment to center secondary source aperture (SSA).

\end{enumerate}

\end{description}


\subsection{Focusing the K\sphinxhyphen{}B Mirrors}
\label{\detokenize{staff:focusing-the-k-b-mirrors}}\begin{itemize}
\item {} \begin{description}
\item[{These are the complete instructions for focusing the K\sphinxhyphen{}B mirrors. Some steps can be skipped if the optics are already aligned and the goal is to tweak the optics.}] \leavevmode\begin{enumerate}
\sphinxsetlistlabels{\arabic}{enumi}{enumii}{}{.}%
\item {} 
\sphinxAtStartPar
Check that the local bump is at the nominal values.

\item {} 
\sphinxAtStartPar
Open the slits: JJ Slits (2.0 x 2.0 mm), SSA (1.0 x 0.05 mm).

\item {} 
\sphinxAtStartPar
Move the K\sphinxhyphen{}B mirrors out of the beam. They should return to 0 pitch and translate out of the beam path.

\item {} 
\sphinxAtStartPar
Make sure the X\sphinxhyphen{}ray beam goes through the system. Check the X\sphinxhyphen{}ray eye. The ion chambers should see X\sphinxhyphen{}rays. The X\sphinxhyphen{}rays should pass through the nanoKB chamber. The X\sphinxhyphen{}ray beam should be about 1.5 x 1.2 mm (HxV) on the Merlin detector. Be sure to keep the total counts below 100k.

\item {} 
\sphinxAtStartPar
Check that the JJ slits are centered on the X\sphinxhyphen{}ray beam. Close down the JJ slits to 0.3 x 0.6 mm (HxV).

\item {} \begin{description}
\item[{Move in and roughly align the K\sphinxhyphen{}B mirrors:}] \leavevmode\begin{itemize}
\item {} 
\sphinxAtStartPar
Start with the fine pitch motors for both K\sphinxhyphen{}B mirrors at 15 μm (the middle of their range).

\item {} 
\sphinxAtStartPar
Move in the vertical mirror. Check that the mirror is flat and set to zero. Move to the middle of the X\sphinxhyphen{}ray beam.

\item {} 
\sphinxAtStartPar
Move in the horizontal mirror. Check that the mirror is flat and set to zero. Move to middle of the X\sphinxhyphen{}ray beam.

\item {} 
\sphinxAtStartPar
Pitch the vertical mirror to 3 mrad. Translate the mirror down by 0.63 mm.

\item {} 
\sphinxAtStartPar
Pitch the horizontal mirror to 3 mrad. Translate the mirror outboard by 0.15 mm.

\item {} 
\sphinxAtStartPar
Check that the focused beam can be seen by the Merlin and the VLM is not blocking the focused beam. VLM positions June 2021 (X, Y, Z) = (\sphinxhyphen{}0.090, \sphinxhyphen{}0.380, \sphinxhyphen{}1.357)

\end{itemize}

\end{description}

\item {} 
\sphinxAtStartPar
Put in the diving board. Look for the fiducial marker patterns (Pt/Cr, 50 nm) with 5 μm wide horizontal and vertical features on the very edge.

\item {} 
\sphinxAtStartPar
Use the VLM and fluorescence signal to roughly align the X\sphinxhyphen{}ray position cross\sphinxhyphen{}hair.

\item {} \begin{description}
\item[{Start with the vertical focus alignment:}] \leavevmode\begin{itemize}
\item {} 
\sphinxAtStartPar
Run a knife\sphinxhyphen{}edge scan across a line to get an initial beam size. \sphinxcode{\sphinxupquote{RE(nano\_knife\_edge(nano\_stage.sy, \sphinxhyphen{}10, 10, 0.2, 0.1))}}

\item {} 
\sphinxAtStartPar
If the beam size is greater than 1 μm, move the coarse Z by 500 μm and look for a smaller beam size. Be aware line features will move horizontally when changing coarse Z.

\item {} 
\sphinxAtStartPar
Repeat until the beam size is smaller than 1 μm.

\item {} 
\sphinxAtStartPar
Run the slit\sphinxhyphen{}scan script. Here we as scanning the sample from \sphinxhyphen{}8 to 8 μm to move across the Pt line. The JJ slits are set to a gap of 0.1 mm and scanned a total of 1 mm centered around the beam center. Some of the knife\sphinxhyphen{}edge scans will not hit the mirror, so these scans will need to be excluded from the final analysis. \sphinxcode{\sphinxupquote{RE(slit\_nanoflyscan(nano\_stage.sy, \sphinxhyphen{}8, 8, 0.1, 0.05, jjslits.v\_trans, 3.01, 4.01, 0.1, jjslits.v\_gap, 0.1))}}

\item {} 
\sphinxAtStartPar
Run the calculation for alignment. \sphinxcode{\sphinxupquote{slit\_nanoflyscan(scan\_id\_list=np.linspace(startid, stopid, numscans), slit\_range=np.linspace(jjslit\_start, jjslit\_stop, numscans), interp\_range={[}2, 3, 4, 5, 6, 7, 8, 9, 10{]}, orthogonality=False)}}

\item {} 
\sphinxAtStartPar
The script will show a plot of the Pt line center and report some values. In particular, pay attention to the defocus amount. Move the sample by the defocus amount using the coarse Z stage.

\item {} 
\sphinxAtStartPar
Run another knife\sphinxhyphen{}edge scan to make sure the focus improved.

\item {} 
\sphinxAtStartPar
Run the slit\sphinxhyphen{}scan script and calculation again. Hopefully upon calculation, the defocus amount is small (\textless{} 100 μm) and the curve is relatively flat. In that case, change the orthogonality flag to True and run the calculation again. Otherwise, repeat until the defocus amount is small.

\item {} 
\sphinxAtStartPar
With orthogonality True, the fine vertical pitch is adjusted. Move the fine pitch actuator for the vertical pitch. Move the coarse Z stage as well.

\item {} 
\sphinxAtStartPar
Run a knife\sphinxhyphen{}edge scan to check the focus improved.

\item {} 
\sphinxAtStartPar
Repeat the slit\sphinxhyphen{}scan and knife\sphinxhyphen{}edge scans with orthogonality True until the focus is acceptable.

\end{itemize}

\end{description}

\item {} \begin{description}
\item[{Focus the horizontal K\sphinxhyphen{}B mirror}] \leavevmode\begin{itemize}
\item {} 
\sphinxAtStartPar
Attenion! For horizontal mirror alignment, only horizontal mirror pitch should be moved to prevent astigmatism in the two focal planes.

\item {} 
\sphinxAtStartPar
Find a line for scanning and run a knife\sphinxhyphen{}edge scan to get the initial beam size.

\item {} 
\sphinxAtStartPar
Run the slit\sphinxhyphen{}scan to scan the JJ slits across the mirror.

\item {} 
\sphinxAtStartPar
Perform the slit\sphinxhyphen{}scan calculations with orthogonality False.

\item {} 
\sphinxAtStartPar
The calculation will output an amount to move the horizontal K\sphinxhyphen{}B mirror in mrad. To translate this to a linear distance for the fine actuator, multiply by 100 mm. Move the horizontal fine pitch actuator by this amount.

\item {} 
\sphinxAtStartPar
Similar to the vertical mirror, run a knife\sphinxhyphen{}edge scan to make sure the actuator was moved the correct direction and measure the new focus.

\item {} 
\sphinxAtStartPar
Repeat the slit\sphinxhyphen{}scans until the focus is acceptable.

\item {} 
\sphinxAtStartPar
Check the horizontal focus as a function of SSA width.

\end{itemize}

\end{description}

\end{enumerate}

\end{description}

\end{itemize}


\subsection{Calibrating the monochromator}
\label{\detokenize{staff:calibrating-the-monochromator}}\begin{description}
\item[{\sphinxstyleemphasis{Calibrating the monochromator is done by collecting XANES spectra across several element absorption edges. A least\sphinxhyphen{}squares fitting routine will then calculate the HDCM parameters for the calibration}}] \leavevmode\begin{enumerate}
\sphinxsetlistlabels{\arabic}{enumi}{enumii}{}{.}%
\item {} 
\sphinxAtStartPar
Collect XANES scans at 3\sphinxhyphen{}5 different energies. For the best fit, a wide range of energies is best. Typically, scans are performed using V, Cr, Fe, Cu, Se, Zr foils. \sphinxstyleemphasis{It is a good idea to record the C1 Roll and C2 Pitch values for each energy. These can be used for a lookup table to improve the peakup function.}:

\begin{sphinxVerbatim}[commandchars=\\\{\}]
\PYG{n}{Bluesky}\PYG{n+nd}{@SRX} \PYG{p}{[}\PYG{l+m+mi}{1}\PYG{p}{]} \PYG{n}{X} \PYG{o}{=} \PYG{n}{getbindingE}\PYG{p}{(}\PYG{l+s+s1}{\PYGZsq{}}\PYG{l+s+s1}{Fe}\PYG{l+s+s1}{\PYGZsq{}}\PYG{p}{)}
\PYG{n}{Bluesky}\PYG{n+nd}{@SRX} \PYG{p}{[}\PYG{l+m+mi}{2}\PYG{p}{]} \PYG{o}{\PYGZpc{}}\PYG{n}{mov} \PYG{n}{energy} \PYG{n}{X}
\PYG{n}{Bluesky}\PYG{n+nd}{@SRX} \PYG{p}{[}\PYG{l+m+mi}{3}\PYG{p}{]} \PYG{n}{RE}\PYG{p}{(}\PYG{n}{peakup\PYGZus{}fine}\PYG{p}{(}\PYG{p}{)}\PYG{p}{)}
\PYG{n}{Bluesky}\PYG{n+nd}{@SRX} \PYG{p}{[}\PYG{l+m+mi}{4}\PYG{p}{]} \PYG{n}{RE}\PYG{p}{(}\PYG{n}{xanes\PYGZus{}plan}\PYG{p}{(}\PYG{p}{[}\PYG{n}{X}\PYG{o}{\PYGZhy{}}\PYG{l+m+mi}{50}\PYG{p}{,} \PYG{n}{X}\PYG{o}{+}\PYG{l+m+mi}{50}\PYG{p}{]}\PYG{p}{,} \PYG{p}{[}\PYG{l+m+mi}{1}\PYG{p}{]}\PYG{p}{,} \PYG{l+m+mf}{0.1}\PYG{p}{)}\PYG{p}{)}
\end{sphinxVerbatim}

\item {} 
\sphinxAtStartPar
Define a dictionary in bluesky with element symbols mapped to scan IDs.:

\begin{sphinxVerbatim}[commandchars=\\\{\}]
\PYG{n}{Bluesky}\PYG{n+nd}{@SRX} \PYG{p}{[}\PYG{l+m+mi}{5}\PYG{p}{]} \PYG{n}{scanlogDic} \PYG{o}{=} \PYG{p}{\PYGZob{}}\PYG{l+s+s1}{\PYGZsq{}}\PYG{l+s+s1}{V}\PYG{l+s+s1}{\PYGZsq{}} \PYG{p}{:} \PYG{l+m+mi}{1000}\PYG{p}{,}
                              \PYG{l+s+s1}{\PYGZsq{}}\PYG{l+s+s1}{Cr}\PYG{l+s+s1}{\PYGZsq{}}\PYG{p}{:} \PYG{l+m+mi}{1001}\PYG{p}{,}
                              \PYG{l+s+s1}{\PYGZsq{}}\PYG{l+s+s1}{Fe}\PYG{l+s+s1}{\PYGZsq{}}\PYG{p}{:} \PYG{l+m+mi}{1002}\PYG{p}{,}
                              \PYG{l+s+s1}{\PYGZsq{}}\PYG{l+s+s1}{Cu}\PYG{l+s+s1}{\PYGZsq{}}\PYG{p}{:} \PYG{l+m+mi}{1003}\PYG{p}{,}
                              \PYG{l+s+s1}{\PYGZsq{}}\PYG{l+s+s1}{Se}\PYG{l+s+s1}{\PYGZsq{}}\PYG{p}{:} \PYG{l+m+mi}{1004}\PYG{p}{,}
                              \PYG{l+s+s1}{\PYGZsq{}}\PYG{l+s+s1}{Zr}\PYG{l+s+s1}{\PYGZsq{}}\PYG{p}{:} \PYG{l+m+mi}{1005}\PYG{p}{\PYGZcb{}}
\end{sphinxVerbatim}

\item {} 
\sphinxAtStartPar
Run the \sphinxstyleemphasis{braggcalib()} function with the dictionary as input. The function will go through each scan and display a plot marking where the edge was found. Finally, this will output the new HDCM parameters.:

\begin{sphinxVerbatim}[commandchars=\\\{\}]
\PYG{n}{Bluesky}\PYG{n+nd}{@SRX} \PYG{p}{[}\PYG{l+m+mi}{6}\PYG{p}{]} \PYG{n}{braggcalib}\PYG{p}{(}\PYG{n}{scanlogDic}\PYG{o}{=}\PYG{n}{scanlogDic}\PYG{p}{,} \PYG{n}{use\PYGZus{}xrf}\PYG{o}{=}\PYG{k+kc}{True}\PYG{p}{)}
\end{sphinxVerbatim}

\item {} 
\sphinxAtStartPar
Update the values in the bluesky profile (10\sphinxhyphen{}machine.py). Save and restart bluesky.

\end{enumerate}

\end{description}


\section{Beamline Maintenance}
\label{\detokenize{staff:beamline-maintenance}}
\begin{sphinxadmonition}{note}{\label{\detokenize{staff:id3}}Todo:}\begin{itemize}
\item {} 
\sphinxAtStartPar
Calibrating the Xspress3

\item {} 
\sphinxAtStartPar
Power loss preparation and recovery

\end{itemize}
\end{sphinxadmonition}


\subsection{Cryocooler}
\label{\detokenize{staff:cryocooler}}
\sphinxAtStartPar
\sphinxstyleemphasis{The manual for the cryocooler can be found here.}

\begin{sphinxadmonition}{note}{\label{\detokenize{staff:id4}}Todo:}\begin{itemize}
\item {} 
\sphinxAtStartPar
Upload cryocooler manual

\end{itemize}
\end{sphinxadmonition}


\subsubsection{Warming the cryocooler}
\label{\detokenize{staff:warming-the-cryocooler}}\begin{enumerate}
\sphinxsetlistlabels{\arabic}{enumi}{enumii}{}{.}%
\item {} 
\sphinxAtStartPar
Connect a turbo\sphinxhyphen{}pump station to the monochromator tank. Pump the turbo so that it reads a pressure in the 10$^{\text{\sphinxhyphen{}8}}$ Torr range.

\item {} 
\sphinxAtStartPar
Close the beamline gate valves to isolate the monochromator.

\item {} 
\sphinxAtStartPar
Open the manual valve between the monochromator and turbo\sphinxhyphen{}pump.

\item {} 
\sphinxAtStartPar
Turn off the ion pump. This should automatically put the cryocooler in “Stop” mode.

\item {} 
\sphinxAtStartPar
The cryocooler will warm up over several days.

\end{enumerate}


\subsubsection{Cooling the cryocooler}
\label{\detokenize{staff:cooling-the-cryocooler}}\begin{enumerate}
\sphinxsetlistlabels{\arabic}{enumi}{enumii}{}{.}%
\item {} 
\sphinxAtStartPar
Verify the monochromator cold cathode gauge is on and the cryocooler is not inhibitted. This typically means a pressure reading better than 10$^{\text{\sphinxhyphen{}7}}$ Torr.

\item {} \begin{description}
\item[{Purge the system according to the cryocooler manual, section 3.3.1, on page 28. \sphinxstyleemphasis{Note: V10 and V11 are variable values, 0\% = Close, 100\% = Open.}}] \leavevmode\begin{itemize}
\item {} 
\sphinxAtStartPar
Verify N$_{\text{2}}$ gas source is at a pressure between 1.5 and 3.0 bar.

\item {} 
\sphinxAtStartPar
Verify the monochromator ion pump and cold cathode gauge are on.

\item {} 
\sphinxAtStartPar
Close all the valves.

\item {} 
\sphinxAtStartPar
Open V9, V10, V20, and V21 fully. Purge for 30 min.

\item {} 
\sphinxAtStartPar
Close V9 and open V11. Purge for 15 min.

\item {} 
\sphinxAtStartPar
Close V11. Open V17 and purge for 15 min.

\item {} 
\sphinxAtStartPar
Close all the valves.

\end{itemize}

\end{description}

\item {} \begin{description}
\item[{Following the cryocooler manual, section 4.2.1.1, fill the sub\sphinxhyphen{}cooler to 15\% and fill the heater vessel to 20\%.}] \leavevmode\begin{itemize}
\item {} 
\sphinxAtStartPar
Verify the liquid N$_{\text{2}}$ source valve is open.

\item {} 
\sphinxAtStartPar
Open V19 to start filling the sub\sphinxhyphen{}cooler.

\item {} 
\sphinxAtStartPar
Close V19 when the sub\sphinxhyphen{}cooler reaches 15\%.

\item {} 
\sphinxAtStartPar
Open heater vessel valve to start filling the heater vessel.

\item {} 
\sphinxAtStartPar
Close heater vessel valve when level reaches 20\%.

\end{itemize}

\end{description}

\item {} \begin{description}
\item[{Follow the automatic cool down proceedure from the manual, section 4.3.}] \leavevmode\begin{itemize}
\item {} 
\sphinxAtStartPar
Verify the liquid N$_{\text{2}}$ source valve is open.

\item {} 
\sphinxAtStartPar
From the cryocooler CSS page, click “Cool Down”.

\item {} 
\sphinxAtStartPar
Once full, in CSS click on the “A” to enable automatic filling of the cryocooler.

\end{itemize}

\end{description}

\end{enumerate}


\section{Controls}
\label{\detokenize{staff:controls}}

\subsection{IOC Monitoring}
\label{\detokenize{staff:ioc-monitoring}}\begin{description}
\item[{On a Debian server, the manage\sphinxhyphen{}iocs tool can be used to monitor the IOC status. SSH into the server that hosts the IOC (\sphinxstyleemphasis{e.g.} xf05idd\sphinxhyphen{}ioc1) and run:}] \leavevmode\begin{itemize}
\item {} 
\sphinxAtStartPar
List all IOCs

\begin{sphinxVerbatim}[commandchars=\\\{\}]
\PYGZdl{} manage\PYGZhy{}iocs report
\end{sphinxVerbatim}

\item {} 
\sphinxAtStartPar
Show IOC status

\begin{sphinxVerbatim}[commandchars=\\\{\}]
\PYGZdl{} manage\PYGZhy{}iocs status
\end{sphinxVerbatim}

\item {} 
\sphinxAtStartPar
Start IOC, \sphinxstyleemphasis{softioc\sphinxhyphen{}example}. The path to the IOC can be found using \sphinxcode{\sphinxupquote{manage\sphinxhyphen{}iocs}}.

\begin{sphinxVerbatim}[commandchars=\\\{\}]
\PYGZdl{} sudo /etc/init.d/softioc\PYGZhy{}example start
\end{sphinxVerbatim}

\item {} 
\sphinxAtStartPar
Stop IOC, \sphinxstyleemphasis{softioc\sphinxhyphen{}example}. The path to the IOC can be found using \sphinxcode{\sphinxupquote{manage\sphinxhyphen{}iocs}}.

\begin{sphinxVerbatim}[commandchars=\\\{\}]
\PYGZdl{} sudo /etc/init.d/softioc\PYGZhy{}example stop
\end{sphinxVerbatim}

\end{itemize}

\item[{On a CentOS server, the IOCs are managed using procServ. This is typically a simple executable script that will start them.}] \leavevmode\begin{itemize}
\item {} 
\sphinxAtStartPar
SSH into the camera server, xf05idd\sphinxhyphen{}ioc2.

\item {} 
\sphinxAtStartPar
To start the IOC for the Blackfly camera

\begin{sphinxVerbatim}[commandchars=\\\{\}]
\PYGZdl{} cd /epics/iocs/cam\PYGZhy{}bfly1
\PYGZdl{} ./start\PYGZus{}cam\PYGZus{}bfly1
\end{sphinxVerbatim}

\item {} 
\sphinxAtStartPar
Using these commands, the IOC will start and you will be in a telnet of the IOC.

\item {} 
\sphinxAtStartPar
To exit the telnet, type \sphinxcode{\sphinxupquote{Ctrl+{]}}} and then \sphinxcode{\sphinxupquote{q}}.

\item {} 
\sphinxAtStartPar
To stop the IOC, the process for procServ must be stopped. The process ID is the second column.

\begin{sphinxVerbatim}[commandchars=\\\{\}]
\PYGZdl{} ps aux | grep procServ
akiss      820  0.0  0.0  27448   976 ?        Ss    2020  29:07 procServ \PYGZhy{}\PYGZhy{}logstamp \PYGZhy{}n cam\PYGZhy{}bfly1 \PYGZhy{}i \PYGZca{}D \PYGZhy{}L /epics/iocs/cam\PYGZhy{}bfly1/log/cam\PYGZhy{}bfly1.log 20001 ./st.cmd
\PYGZdl{} sudo kill \PYGZhy{}9 820
\end{sphinxVerbatim}

\end{itemize}

\end{description}


\subsection{Motion Controls}
\label{\detokenize{staff:motion-controls}}
\begin{sphinxadmonition}{note}{\label{\detokenize{staff:id5}}Todo:}\begin{itemize}
\item {} 
\sphinxAtStartPar
Insert table with: Motor controller, IOC, Motor, PV, Bluesky object

\item {} 
\sphinxAtStartPar
Rearrange table to be motor, bluesky, IOC, controller, PV?

\item {} 
\sphinxAtStartPar
List of all IOCs on each server

\end{itemize}
\end{sphinxadmonition}


\begin{savenotes}\sphinxattablestart
\raggedright
\sphinxcapstartof{table}
\sphinxthecaptionisattop
\sphinxcaption{xf05ida\sphinxhyphen{}ioc1 motors}\label{\detokenize{staff:xf05ida-ioc1-motors}}
\sphinxaftertopcaption
\begin{tabulary}{\linewidth}[t]{|T|T|T|T|T|}
\hline
\sphinxstyletheadfamily 
\sphinxAtStartPar
Motor Controller
&\sphinxstyletheadfamily 
\sphinxAtStartPar
IOC
&\sphinxstyletheadfamily 
\sphinxAtStartPar
Motor
&\sphinxstyletheadfamily 
\sphinxAtStartPar
PV
&\sphinxstyletheadfamily 
\sphinxAtStartPar
Bluesky Object
\\
\hline
\sphinxAtStartPar
mc01
&
\sphinxAtStartPar
softioc\sphinxhyphen{}mc01
&
\sphinxAtStartPar
testmotor
&
\sphinxAtStartPar
XF:
&
\sphinxAtStartPar
bs.motor
\\
\hline
\end{tabulary}
\par
\sphinxattableend\end{savenotes}


\begin{savenotes}\sphinxattablestart
\raggedright
\sphinxcapstartof{table}
\sphinxthecaptionisattop
\sphinxcaption{xf05idd\sphinxhyphen{}ioc1 motors}\label{\detokenize{staff:xf05idd-ioc1-motors}}
\sphinxaftertopcaption
\begin{tabulary}{\linewidth}[t]{|T|T|T|T|T|}
\hline
\sphinxstyletheadfamily 
\sphinxAtStartPar
Motor Controller
&\sphinxstyletheadfamily 
\sphinxAtStartPar
IOC
&\sphinxstyletheadfamily 
\sphinxAtStartPar
Motor
&\sphinxstyletheadfamily 
\sphinxAtStartPar
PV
&\sphinxstyletheadfamily 
\sphinxAtStartPar
Bluesky Object
\\
\hline
\sphinxAtStartPar
mc01
&
\sphinxAtStartPar
softioc\sphinxhyphen{}mc01
&
\sphinxAtStartPar
testmotor
&
\sphinxAtStartPar
XF:
&
\sphinxAtStartPar
bs.motor
\\
\hline
\end{tabulary}
\par
\sphinxattableend\end{savenotes}


\begin{savenotes}\sphinxattablestart
\raggedright
\sphinxcapstartof{table}
\sphinxthecaptionisattop
\sphinxcaption{xf05idd\sphinxhyphen{}ioc3 motors}\label{\detokenize{staff:xf05idd-ioc3-motors}}
\sphinxaftertopcaption
\begin{tabulary}{\linewidth}[t]{|T|T|T|T|T|}
\hline
\sphinxstyletheadfamily 
\sphinxAtStartPar
Motor
&\sphinxstyletheadfamily 
\sphinxAtStartPar
Bluesky Object
&\sphinxstyletheadfamily 
\sphinxAtStartPar
Motor Controller
&\sphinxstyletheadfamily 
\sphinxAtStartPar
IOC
&\sphinxstyletheadfamily 
\sphinxAtStartPar
PV
\\
\hline
\sphinxAtStartPar
nanoKBv angle calc
&
\sphinxAtStartPar
bs.motor
&
\sphinxAtStartPar
none
&
\sphinxAtStartPar
softioc\sphinxhyphen{}anglecalc
&
\sphinxAtStartPar
XF:05IDD\sphinxhyphen{}ES:1\{nKB:vert\sphinxhyphen{}Ax:PC\}Mtr
\\
\hline
\sphinxAtStartPar
nanoKBh angle calc
&
\sphinxAtStartPar
bs.motor
&
\sphinxAtStartPar
none
&
\sphinxAtStartPar
softioc\sphinxhyphen{}anglecalc
&
\sphinxAtStartPar
XF:05IDD\sphinxhyphen{}ES:1\{nKB:horz\sphinxhyphen{}Ax:PC\}Mtr
\\
\hline
\sphinxAtStartPar
testmotor
&
\sphinxAtStartPar
bs.motor
&
\sphinxAtStartPar
fpsensor1
&
\sphinxAtStartPar
softioc\sphinxhyphen{}fpsensor
&
\sphinxAtStartPar
XF:05IDD\sphinxhyphen{}ES:1\{FPS:1\sphinxhyphen{}Chan0\}Pos\sphinxhyphen{}I
\\
\hline
\sphinxAtStartPar
testmotor
&
\sphinxAtStartPar
bs.motor
&
\sphinxAtStartPar
fpsensor1
&
\sphinxAtStartPar
softioc\sphinxhyphen{}fpsensor
&
\sphinxAtStartPar
XF:05IDD\sphinxhyphen{}ES:1\{FPS:1\sphinxhyphen{}Chan1\}Pos\sphinxhyphen{}I
\\
\hline
\sphinxAtStartPar
testmotor
&
\sphinxAtStartPar
bs.motor
&
\sphinxAtStartPar
fpsensor1
&
\sphinxAtStartPar
softioc\sphinxhyphen{}fpsensor
&
\sphinxAtStartPar
XF:05IDD\sphinxhyphen{}ES:1\{FPS:1\sphinxhyphen{}Chan2\}Pos\sphinxhyphen{}I
\\
\hline
\sphinxAtStartPar
nanoKBv Fine Pitch
&
\sphinxAtStartPar
bs.motor
&
\sphinxAtStartPar
PI E518
&
\sphinxAtStartPar
softioc\sphinxhyphen{}mcd19
&
\sphinxAtStartPar
XF:05IDD\sphinxhyphen{}ES:1\{nKB:vert\sphinxhyphen{}Ax:PFPI\}Mtr
\\
\hline
\sphinxAtStartPar
nanoKBh Fine Pitch
&
\sphinxAtStartPar
bs.motor
&
\sphinxAtStartPar
PI E518
&
\sphinxAtStartPar
softioc\sphinxhyphen{}mcd19
&
\sphinxAtStartPar
XF:05IDD\sphinxhyphen{}ES:1\{nKB:horz\sphinxhyphen{}Ax:PFPI\}Mtr
\\
\hline
\sphinxAtStartPar
nanoKBh Coarse Pitch
&
\sphinxAtStartPar
bs.motor
&
\sphinxAtStartPar
PI E712
&
\sphinxAtStartPar
softioc\sphinxhyphen{}mcd20
&
\sphinxAtStartPar
XF:05IDD\sphinxhyphen{}ES:1\{nKB:horz\sphinxhyphen{}Ax:PC\}Mtr
\\
\hline
\sphinxAtStartPar
nanoKBv Coarse Pitch
&
\sphinxAtStartPar
bs.motor
&
\sphinxAtStartPar
PI E712
&
\sphinxAtStartPar
softioc\sphinxhyphen{}mcd24
&
\sphinxAtStartPar
XF:05IDD\sphinxhyphen{}ES:1\{nKB:vert\sphinxhyphen{}Ax:PC\}Mtr
\\
\hline
\sphinxAtStartPar
Sample Coarse Z
&
\sphinxAtStartPar
nano\_stage.z
&
\sphinxAtStartPar
Smaract
&
\sphinxAtStartPar
softioc\sphinxhyphen{}mcd26
&
\sphinxAtStartPar
XF:05IDD\sphinxhyphen{}ES:1\{nKB:Smpl\sphinxhyphen{}Ax:sz\}Mtr
\\
\hline
\sphinxAtStartPar
Sample Coarse X
&
\sphinxAtStartPar
nano\_stage.x
&
\sphinxAtStartPar
Smaract
&
\sphinxAtStartPar
softioc\sphinxhyphen{}mcd26
&
\sphinxAtStartPar
XF:05IDD\sphinxhyphen{}ES:1\{nKB:Smpl\sphinxhyphen{}Ax:sx\}Mtr
\\
\hline
\sphinxAtStartPar
Sample Coarse Y
&
\sphinxAtStartPar
nano\_stage.y
&
\sphinxAtStartPar
Smaract
&
\sphinxAtStartPar
softioc\sphinxhyphen{}mcd26
&
\sphinxAtStartPar
XF:05IDD\sphinxhyphen{}ES:1\{nKB:Smpl\sphinxhyphen{}Ax:sy\}Mtr
\\
\hline
\sphinxAtStartPar
Sample Theta
&
\sphinxAtStartPar
nano\_stage.th
&
\sphinxAtStartPar
Smaract
&
\sphinxAtStartPar
softioc\sphinxhyphen{}mcd26
&
\sphinxAtStartPar
XF:05IDD\sphinxhyphen{}ES:1\{nKB:Smpl\sphinxhyphen{}Ax:th\}Mtr
\\
\hline
\sphinxAtStartPar
Sample Top Z
&
\sphinxAtStartPar
nano\_stage.topx
&
\sphinxAtStartPar
Smaract
&
\sphinxAtStartPar
softioc\sphinxhyphen{}mcd26
&
\sphinxAtStartPar
XF:05IDD\sphinxhyphen{}ES:1\{nKB:Smpl\sphinxhyphen{}Ax:zth\}Mtr
\\
\hline
\sphinxAtStartPar
Sample Top X
&
\sphinxAtStartPar
nano\_stage.topz
&
\sphinxAtStartPar
Smaract
&
\sphinxAtStartPar
softioc\sphinxhyphen{}mcd26
&
\sphinxAtStartPar
XF:05IDD\sphinxhyphen{}ES:1\{nKB:Smpl\sphinxhyphen{}Ax:xth\}Mtr
\\
\hline
\sphinxAtStartPar
Sample Scanner X
&
\sphinxAtStartPar
nano\_stage.sx
&
\sphinxAtStartPar
nPoint
&
\sphinxAtStartPar
softioc\sphinxhyphen{}nPoint
&
\sphinxAtStartPar
XF:05IDD\sphinxhyphen{}ES:1\{nKB:Smpl\sphinxhyphen{}Ax:ssx\}Mtr
\\
\hline
\sphinxAtStartPar
Sample Scanner Y
&
\sphinxAtStartPar
nano\_stage.sy
&
\sphinxAtStartPar
nPoint
&
\sphinxAtStartPar
softioc\sphinxhyphen{}nPoint
&
\sphinxAtStartPar
XF:05IDD\sphinxhyphen{}ES:1\{nKB:Smpl\sphinxhyphen{}Ax:ssy\}Mtr
\\
\hline
\sphinxAtStartPar
Sample Scanner Z
&
\sphinxAtStartPar
nano\_stage.sz
&
\sphinxAtStartPar
nPoint
&
\sphinxAtStartPar
softioc\sphinxhyphen{}nPoint
&
\sphinxAtStartPar
XF:05IDD\sphinxhyphen{}ES:1\{nKB:Smpl\sphinxhyphen{}Ax:ssz\}Mtr
\\
\hline
\sphinxAtStartPar
testmotor
&
\sphinxAtStartPar
bs.motor
&
\sphinxAtStartPar
picoscale
&
\sphinxAtStartPar
softioc\sphinxhyphen{}picoscale
&
\sphinxAtStartPar
XF:
\\
\hline
\sphinxAtStartPar
nanoZebra
&
\sphinxAtStartPar
nanoZebra
&
\sphinxAtStartPar
none
&
\sphinxAtStartPar
softioc\sphinxhyphen{}zebra
&
\sphinxAtStartPar
XF:05IDD\sphinxhyphen{}ES:1\{Dev:Zebra2\}
\\
\hline
\end{tabulary}
\par
\sphinxattableend\end{savenotes}


\begin{savenotes}\sphinxattablestart
\raggedright
\sphinxcapstartof{table}
\sphinxthecaptionisattop
\sphinxcaption{xf05idd\sphinxhyphen{}det1}\label{\detokenize{staff:xf05idd-det1}}
\sphinxaftertopcaption
\begin{tabulary}{\linewidth}[t]{|T|T|T|T|T|}
\hline
\sphinxstyletheadfamily 
\sphinxAtStartPar
Motor Controller
&\sphinxstyletheadfamily 
\sphinxAtStartPar
IOC
&\sphinxstyletheadfamily 
\sphinxAtStartPar
Motor
&\sphinxstyletheadfamily 
\sphinxAtStartPar
PV
&\sphinxstyletheadfamily 
\sphinxAtStartPar
Bluesky Object
\\
\hline
\sphinxAtStartPar
mc01
&
\sphinxAtStartPar
softioc\sphinxhyphen{}mc01
&
\sphinxAtStartPar
testmotor
&
\sphinxAtStartPar
XF:
&
\sphinxAtStartPar
bs.motor
\\
\hline
\end{tabulary}
\par
\sphinxattableend\end{savenotes}


\begin{savenotes}\sphinxattablestart
\raggedright
\sphinxcapstartof{table}
\sphinxthecaptionisattop
\sphinxcaption{xf05idd\sphinxhyphen{}det2}\label{\detokenize{staff:xf05idd-det2}}
\sphinxaftertopcaption
\begin{tabulary}{\linewidth}[t]{|T|T|T|T|T|}
\hline
\sphinxstyletheadfamily 
\sphinxAtStartPar
Motor Controller
&\sphinxstyletheadfamily 
\sphinxAtStartPar
IOC
&\sphinxstyletheadfamily 
\sphinxAtStartPar
Motor
&\sphinxstyletheadfamily 
\sphinxAtStartPar
PV
&\sphinxstyletheadfamily 
\sphinxAtStartPar
Bluesky Object
\\
\hline
\sphinxAtStartPar
mc01
&
\sphinxAtStartPar
softioc\sphinxhyphen{}mc01
&
\sphinxAtStartPar
testmotor
&
\sphinxAtStartPar
XF:
&
\sphinxAtStartPar
bs.motor
\\
\hline
\end{tabulary}
\par
\sphinxattableend\end{savenotes}


\begin{savenotes}\sphinxattablestart
\raggedright
\sphinxcapstartof{table}
\sphinxthecaptionisattop
\sphinxcaption{xf05idd\sphinxhyphen{}ioc2}\label{\detokenize{staff:xf05idd-ioc2}}
\sphinxaftertopcaption
\begin{tabulary}{\linewidth}[t]{|T|T|T|T|T|}
\hline
\sphinxstyletheadfamily 
\sphinxAtStartPar
Motor Controller
&\sphinxstyletheadfamily 
\sphinxAtStartPar
IOC
&\sphinxstyletheadfamily 
\sphinxAtStartPar
Motor
&\sphinxstyletheadfamily 
\sphinxAtStartPar
PV
&\sphinxstyletheadfamily 
\sphinxAtStartPar
Bluesky Object
\\
\hline
\sphinxAtStartPar
mc01
&
\sphinxAtStartPar
softioc\sphinxhyphen{}mc01
&
\sphinxAtStartPar
testmotor
&
\sphinxAtStartPar
XF:
&
\sphinxAtStartPar
bs.motor
\\
\hline
\end{tabulary}
\par
\sphinxattableend\end{savenotes}


\subsection{EPS}
\label{\detokenize{staff:eps}}
\begin{sphinxadmonition}{note}{\label{\detokenize{staff:id6}}Todo:}\begin{itemize}
\item {} 
\sphinxAtStartPar
Upload wiring diagrams

\end{itemize}
\end{sphinxadmonition}


\chapter{Indices and tables}
\label{\detokenize{index:indices-and-tables}}\begin{itemize}
\item {} 
\sphinxAtStartPar
\DUrole{xref,std,std-ref}{genindex}

\item {} 
\sphinxAtStartPar
\DUrole{xref,std,std-ref}{modindex}

\item {} 
\sphinxAtStartPar
\DUrole{xref,std,std-ref}{search}

\end{itemize}



\renewcommand{\indexname}{Index}
\printindex
\end{document}